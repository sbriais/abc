\documentclass{article}
\usepackage[latin1]{inputenc}
\usepackage{process}

\newcommand{\LAMBDA}[2]{(\lambda {#1}){#2}}

\newenvironment{bnf}{%
  \newif\ifbnftab%
  \bnftabfalse%
  \let\bnfnewline\\%
  \def\IS{%
    &$::=$&%
    \bnftabtrue}
  \def\OR{%
    \ifbnftab%
      \vrule~%
    \else%
      \expandafter\bnfORline%
    \fi%
    \bnftabtrue}%
  \def\bnfORline{&\hfill\vrule&}%
  \def\\{%
    \bnfnewline%
    \bnftabfalse}%
  \def\BRK{\cr&&}%
  \sffamily%
  \par\medskip
  \begin{tabular}{rcll}}{%
  \end{tabular}%
  \par\medskip}


\author{\textsc{S�bastien BRIAIS}}
\title{The ABC User's Guide}

\begin{document}
\maketitle

\begin{abstract}
  In this document, we describe how to use the Another Bisimilarity
  Checker (ABC) which is a tool that checks for open equivalence
  between terms of the \picalculus. Open bisimulation was first
  defined by Sangiorgi \cite{sangiorgi:theory-bisimulation}. The ABC
  implements the equivalence checker of the Mobility Workbench
  \cite{victor:verification-tool}. In the following, we assume the
  reader is familiar with the \picalculus.
  
  The ABC is released under the GPL. The distribution can be downloaded from
  the ABC homepage located at:
  \begin{center}
  \verb�http://lampwww.epfl.ch/~sbriais/abc.html�
\end{center}
\end{abstract}

\section{Input syntax: \picalculus terms}
\subsection{The \picalculus of ABC (readable)}
We assume a countably infinite set of names denoted by
$a,b,x,v,\ldots$ and a disjoint set of agent identifiers denoted by
$X$. The syntax of \picalculus terms is described by the following grammar:
\begin{bnf}
$P,Q$ \IS $\NIL$ 
      \OR $\PAR{P}{Q}$ 
      \OR $\SUM{P}{Q}$ 
      \OR $\PREFIX{\alpha}{P}$
      \OR $\NU{\TUPLE{x}}{P}$
      \OR $\APPLY{X}{\TUPLE{v}}$
      \OR $\MATCH{a}{b}{P}$\\

$\alpha$ \IS $\INPUT{a}{\TUPLE{x}}$ 
         \OR $\OUTPUT{}{a}{\TUPLE{v}}$ 
         \OR $\TAU$
\end{bnf}
As usual, we write $\TUPLE{v}$ as a shortcut for $v_1,\ldots,v_n$ for some $n$.

Sometimes, \picalculus syntax is presented in terms of concretions and
abstractions. A concretion is mainly something like $\CONC{\TUPLE{v}}{C}$
where $C$ is a process or a concretion. An abstraction is mainly
something like $\LAMBDA{\TUPLE{x}}{A}$ where $A$ is a process or an
abstraction. In this setting, prefixes are either $\TAU$, an input
name $\INPUT{a}{}$ or an output name $\OUTPUT{}{a}{}$. The presentation
in terms of concretions and abstractions make the labelled-transition
semantics easier to write.
It is easy to convert classic syntax to this one. Indeed, we have
$$
\begin{array}{rcl}
\PREFIX{\INPUT{a}{\TUPLE{x}}}{P} & \equiv & \PREFIX{\INPUT{a}{}}{\LAMBDA{\TUPLE{x}}{P}} \\
\PREFIX{\OUTPUT{}{a}{\TUPLE{v}}}{P} & \equiv & \PREFIX{\OUTPUT{}{a}{}}{\CONC{\TUPLE{v}}{P}}
\end{array}
$$

The \picalculus terms accepted by the ABC are the one described by the
previous grammar. It also allows the user to enter a \picalculus
term written in terms of abstractions and concretions.

\paragraph{Remark:} 
We sometimes use the word \emph{name} and sometimes the word
\emph{variable}.  However, these two words designate the same concept
because in the case of open bisimulation, the two notions coincides.

\subsection{Input syntax}
\begin{tabular}{cc}
\begin{bnf}
  agent \IS factor\\
  \OR factor \verb:"+": agent\\
  \\
  factor \IS term \\
  \OR term \verb:"|": factor\\
  \\
  term \IS \verb:"0": \\
  \OR ident params \\
  \OR prefix \verb:".": term \\
  \OR \verb:"[": vars \verb:"]": term \\
  \OR \verb:"[": var \verb:"=": var \verb:"]": term \\
  \OR \verb:"(": \verb:"\": vars \verb:")": term \\
  \OR \verb:"(": \verb:"^": vars \verb:")": term \\
  \OR term var \\
  \OR \verb:"(": agent \verb:"):\\
\end{bnf}
&
\begin{bnf}
vars \IS var \\
     \OR var \verb:",": vars \\
\\
prefix \IS \verb:"t": \\
       \OR var params \\
       \OR \verb:"'": var oparams \\
\\
params \IS $\epsilon$ \\
       \OR \verb:"(": vars \verb:")": \\
\\
oparams \IS $\epsilon$ \\
       \OR \verb:"<": vars \verb:">": \\
\\
var \IS [a-z][a-zA-Z\_/.0-9]* \\
\\
ident \IS [A-Z][a-zA-Z\_/.0-9]* \\
\end{bnf}
\end{tabular}

\paragraph{Remark:} It is optional to write the nil process after an action.

\section{Commands of the ABC}
\subsection{Agent definition}
\label{ss:agentdef}
\textbf{agent} {\sffamily ident params = agent}


Note that only \emph{closed agents}\footnote{for efficiency reasons,
  see also subsection \ref{ss:context}} are allowed (i.e.
$\mathrm{fn}$({\sffamily agent}) $\subset${\sffamily params}).

\subsection{Checking (open) equivalence}
An open bisimulation is a family of binary relations between processes
of the \picalculus which is indexed by a set of \emph{distinctions}. A
distinction $D$ is an irreflexive and symmetric binary relation
between names. It represents all the disequalities that should hold.
It can also be used to ``simulate'' free constant names. For instance,
assuming that you have defined an agent $P(a,b,c)$ and that you want
$a$ to be a constant name distinct from each other name, you should
work with the initial distinction $\{(a,b),(a,c),(b,a),(c,a)\}$.

\subsubsection{Strong equivalence}
\textbf{eq} {\sffamily agent1 agent2}\\
\textbf{eqd} {\sffamily (vars) agent1 agent2}


The command \textbf{eq} checks strong equivalence between {\sffamily
  agent1} and {\sffamily agent2}. The command \textbf{eqd} takes into
account the distinction such that each variable in {\sffamily vars} is
distinct from all other free names in {\sffamily agent1} and
{\sffamily agent2}.

\subsubsection{Weak equivalence}
\textbf{weq} {\sffamily agent1 agent2}\\
\textbf{weqd} {\sffamily (vars) agent1 agent2}


Similar to strong equivalence.

\subsection{Viewing agents}
\textbf{print} {\sffamily agent}\\
\textbf{show} {\sffamily agent}


The command \textbf{print} pretty-prints the given {\sffamily agent}
whereas the command \textbf{show} pretty-prints the standard form of
the given {\sffamily agent}.


Recall that every abstraction $F$ can be converted to a standard form
$F \equiv (\lambda \tilde{x})P$ by pushing restrictions inwards and
that every concretion $C$ can be converted to a standard form $C
\equiv (\nu \tilde{y})[\tilde{x}]P$ with $\tilde{y} \subset \tilde{x}$
by pulling restrictions on names in $\tilde{x}$ outwards and pushing
all others inwards.

\subsection{Exploring the behaviour of an agent}
\textbf{step} {\sffamily agent}


The command \textbf{step} enters in an interactive mode in which it is
possible to view and simulate the (strong) commitments of the given
{\sffamily agent}.

\subsection{Setting memory usage}
\textbf{scale} integer\\
\textbf{value} integer

The command \textbf{scale} fixes the scale for setting the memory
usage. The command \textbf{value} sets the memory usage to
$\frac{\textbf{value}}{\textbf{scale}}$. This rational number
represents the frequency for caching computed data.

\subsection{Other commands}
\subsubsection{Help}
\textbf{help}

The command \textbf{help} give the user some help.

\subsubsection{Resetting the ABC}
\textbf{reset}

This command clears all the memory of the ABC.

\subsubsection{Loading a file}
\textbf{load} ``{\sffamily filename}''

This command loads the given file in the ABC.

\subsubsection{Exiting}
\textbf{exit}

\textbf{exit} makes you exit the ABC.


\section{Examples}
\subsection{A beginner's example}
\begin{verbatim}
bash-2.05b$ ./abc.opt
Welcome to Another Bisimulation Checker
\end{verbatim}

After having started the ABC, we define the agents $\FUN{A}$ and $\FUN{B}$ with 
$$
\begin{array}{rcl}
\APPLY{A}{x,y} & \EQDEF & \PAR{\PREFIX{\INPUT{x}{}}{\NIL}}{\PREFIX{\OUTPUT{}{y}{}}{\NIL}} \\
\APPLY{B}{x,y} & \EQDEF & \SUM{\PREFIX{\INPUT{x}{}}{\PREFIX{\OUTPUT{}{y}{}}{\NIL}}}{\PREFIX{\OUTPUT{}{y}{}}{\PREFIX{\INPUT{x}{}}{\NIL}}}
\end{array}
$$
\begin{verbatim}
abc > agent A(x,y) = x.0 | 'y.0
Agent A is defined.
abc > agent B(x,y) = x.'y.0 + 'y.x.0
Agent B is defined.
\end{verbatim}

It is well known that $\APPLY{A}{x,y} \not\sim^{\mathrm{o}}_{\EMPTY}
\APPLY{B}{x,y}$\footnote{$\sim^{\mathrm{o}}_D$ is the open
  bisimilarity with the set of distinctions $D$.}.
\begin{verbatim}
abc > eq A(x,y) B(x,y)
The two agents are not strongly related (2).
Do you want to see a trace of execution (yes/no) ? yes
trace of A x y
 -t-> 0
trace of B x y
 

\end{verbatim}

Indeed, $\APPLY{A}{x,y}$ can perform a $\TAU$ commitment that
$\APPLY{B}{x,y}$ can't (as shown in the trace).
\begin{verbatim}
abc > step A(x,y)
1: { x=y }  =>  A x y --t-> 0

2: {  }  =>  A x y --x-> 'y.0

3: {  }  =>  A x y --'y-> x.0

Please choose a commitment (between 1 and 3) or 0 to exit: 1
no more commitments
abc > step B(x,y)
1: {  }  =>  B x y --x-> 'y.0

2: {  }  =>  B x y --'y-> x.0

Please choose a commitment (between 1 and 2) or 0 to exit: 1
1: {  }  =>  'y.0 --'y-> 0

Please choose a commitment (between 1 and 1) or 0 to exit: 1
no more commitments
\end{verbatim}

\paragraph{Remark:}
We see above that the commitments are shown numbered. Each commitment
has the form:
\begin{verbatim}
a set of conditions => agent --action--> agent
\end{verbatim}
The set of conditions is the one that should be satisfied in order for
the commitment to take place.

\bigskip

Thus, $\APPLY{A}{x,y} \sim^{\mathrm{o}}_{\SET{x \not= y}} \APPLY{B}{x,y}$.
\begin{verbatim}
abc > eqd (x,y) A(x,y) B(x,y)
The two agents are strongly related (3).
Do you want to see the core of the bisimulation (yes/no) ? yes
{
  (
    x.0
    {
      (x, y)
    }
    x.0
  )

  (
    'y.0
    {
      (x, y)
    }
    'y.0
  )

  (
    ('y.x.0 + x.'y.0)
    {
      (x, y)
    }
    ('y.0 | x.0)
  )
}
\end{verbatim}

\paragraph{Remark:}
What we call the \emph{core} of a bisimulation is a ternary relation
between an agent, a set of distinctions and an other agent such that
its symmetric closure plus the identity relation is a bisimulation.

\bigskip

Moreover, if we define an agent $\FUN{C}$ such that
$$
\begin{array}{rcl}
\APPLY{C}{x,y} & \EQDEF & \SUM{\APPLY{B}{x,y}}{\MATCH{x}{y}{\PREFIX{\TAU}{\NIL}}}
\end{array}
$$
\begin{verbatim}
abc > agent C(x,y) = B(x,y) + [x=y]t.0
Agent C is defined.
\end{verbatim}

We have $\APPLY{A}{x,y} \sim^{\mathrm{o}}_{\EMPTY} \APPLY{C}{x,y}$.
\begin{verbatim}
abc > eq A(x,y) C(x,y)
The two agents are strongly related (4).
Do you want to see the core of the bisimulation (yes/no) ? yes
{
  (
    0
    { }
    0
  )

  (
    x.0
    { }
    x.0
  )

  (
    'y.0
    { }
    'y.0
  )

  (
    ([x=y]t.0 + 'y.x.0 + x.'y.0)
    { }
    ('y.0 | x.0)
  )
}
\end{verbatim}

End of this example.
\begin{verbatim}
abc > exit
bash-2.05b$
\end{verbatim}

\subsection{Implicit context}
\label{ss:context}
What about free names? The notion of \emph{implicit context} is our
answer to the limitation explained in Subsection \ref{ss:agentdef}.
It appears often, when defining a system, that several agents use
exactly the same parameters that can be seen as free names on which
the system depends. The following example illustrates this fact:
$$
\begin{array}{rcl}
\FUN{Buf}_{20}(in,out) & = & \PREFIX{\INPUT{in}{x}}{\FUN{Buf}_{21}(in,out,x)} \\
\FUN{Buf}_{21}(in,out,x) & = & \SUM{\PREFIX{\INPUT{in}{y}}{\FUN{Buf}_{22}(in,out,x,y)}}{\PREFIX{\OUTPUT{}{out}{x}}{\FUN{Buf}_{20}(in,out)}} \\
\FUN{Buf}_{22}(in,out,x,y) & = & \PREFIX{\OUTPUT{}{out}{y}}{\FUN{Buf}_{21}(in,out,x)}
\end{array}
$$

If free names were allowed, we would have assumed to have two names
$in$ and $out$ and written:
$$
\begin{array}{rcl}
\FUN{Buf}_{20} & = & \PREFIX{\INPUT{in}{x}}{\FUN{Buf}_{21}(x)} \\
\FUN{Buf}_{21}(x) & = & \SUM{\PREFIX{\INPUT{in}{y}}{\FUN{Buf}_{22}(x,y)}}{\PREFIX{\OUTPUT{}{out}{x}}{\FUN{Buf}_{20}}} \\
\FUN{Buf}_{22}(x,y) & = & \PREFIX{\OUTPUT{}{out}{y}}{\FUN{Buf}_{21}(x)}
\end{array}
$$

This mechanism is precisely the one behind implicit context (actually,
it should be qualified semi-explicit context).

The idea is that a context is a stack of names. Each agent has its own
context. Moreover, there is a current context which is global. When an
agent is not defined, its context is the current one which evolve
dynamically. It is possible to push names in the current context or to
push names out of it. When an agent is defined, its context during his
definition is, as previously said, the current one and it is
duplicated to be its own context. The context of an agent can only
shrink but not grow. So, when a name is pushed in the current context,
the contexts of the defined agents do not evolve whereas when a name
is popped out of the current context, this name is popped out of the
contexts of all the defined agents that contained it. When an agent
$A$ is used in the definition of another agent $B$, the agent $A$ is
partially applied to its context. So, if the context of $A$ was $x,y$
and the one of $B$ was $x,y,z$, when writing $A$ into the definition
of $B$ was understood as $A(x,y)$ (in fact: $A x y$).  We warn the
reader to be very careful when re-loading a file for example (or
redefining an agent). Indeed, since the context of an undefined agent
is the current one and the context of an already defined agent is its
own one, there can happen some strange things when defining two times
a recursive agent (or two mutually recursive agents). That is why
there is a command to undefine an agent.

Here are the added commands to handle implicit contexts:

\mbox{}\\
\textbf{push} {\sffamily var$_1$ ... var$_n$}

This command pushes {\sffamily var$_1$ ... var$_n$} into the
current context.

\mbox{}\\
\textbf{pop} [$n$]

This command pops $n$ variables out of the current context ($n$ is a
positive integer, if $n$ is not specified, then $1$ is the value by
default).

\mbox{}\\
\textbf{implicit}

This command shows the current implicit context.

\mbox{}\\
\textbf{clear} {\sffamily ident$_1$ ... ident$_n$}

This command undefines the agent {\sffamily ident$_1$ ... ident$_n$}
if they were defined.

The next subsection rewrites the example of this subsection in ABC.

\subsubsection{Basic example}
At the beginning, the current implicit context is empty.
\begin{verbatim}
bash-2.05b$ ./abc
Welcome to Another Bisimulation Checker
abc > implicit
No implicit variables.
\end{verbatim}

We first push $in$ and $out$ into the current context.
\begin{verbatim}
abc > push in out
Pushing in out
abc > implicit
Implicit variables are in out
\end{verbatim}

We then define the agent $\FUN{Buf}_{20}$.
\begin{verbatim}
abc > agent Buf20 = in(x).Buf21(x)
Agent Buf20 is defined.
\end{verbatim}

Let us check the standard form of $\FUN{Buf}_{20}$.
\begin{verbatim}
abc > show Buf20
in(x0). Buf21 in out x0
\end{verbatim}

Note that when we write $\FUN{Buf}_{21}$, it really means $\FUN{Buf}_{21}\ in\ out$, since the implicit
context of $\FUN{Buf}_{21}$ is the current one which is $in\ out$ as shown by the command \verb�implicit�.
\begin{verbatim}
abc > implicit
Implicit variables are in out
\end{verbatim}

We continue by defining the agents $\FUN{Buf}_{21}$ and $\FUN{Buf}_{22}$. However,
we can notice that $x$ is also a common additionnal parameter of these
two agents. So, we add $x$ to the current context and then define the agents.

\begin{verbatim}
abc > push x
Pushing x
abc > agent Buf21 = 'out<x>.Buf20 + in(y).Buf22 y
Agent Buf21 is defined.
abc > agent Buf22(y) = 'out<y>.Buf21
Agent Buf22 is defined.
\end{verbatim}

We then display the standard form of the three agents and the current
context.
\begin{verbatim}
abc > implicit
Implicit variables are in out x
abc > show Buf20
in(x0). Buf21 in out x0
abc > show Buf21
( 'out<x>. Buf20 in out +  in(x0). Buf22 in out x x0 )
abc > show Buf22
(\x0) 'out<x0>. Buf21 in out x
\end{verbatim}

We can now pop all the ``free names'' and redo the same actions.
\begin{verbatim}
abc > pop 3
Popping x out in
abc > implicit
No implicit variables.
abc > show Buf20
(\x0, x1) x0(x2). Buf21 x0 x1 x2
abc > show Buf21
(\x0, x1, x2) ( 'x1<x2>. Buf20 x0 x1 +  x0(x3). Buf22 x0 x1 x2 x3 )
abc > show Buf22
(\x0, x1, x2, x3) 'x1<x3>. Buf21 x0 x1 x2
\end{verbatim}

If we prefer, in a more readable setting:
\begin{verbatim}
abc > show Buf20 in out
in(x0). Buf21 in out x0
abc > show Buf21 in out x
( 'out<x>. Buf20 in out +  in(x0). Buf22 in out x x0 )
abc > show Buf22 in out x y
'out<y>. Buf21 in out x
\end{verbatim}

%% \subsubsection{Advanced example}
%% One difficulty that appears with the concept of implicit context is
%% the behavior when there is mutually recursive agent definitions.


%% In the following, we define two mutually recursive agents $A$ and $B$,
%% with $\APPLY{A}{x} = \PREFIX{\INPUT{x}{y}}{\APPLY{B}{x,y}}$ and
%% $\APPLY{B}{x,y} = \PREFIX{\OUTPUT{}{x}{}}{\APPLY{A}{x}}$.
%% \begin{verbatim}
%% bash-2.05b$ ./abc
%% Welcome to Another Bisimulation Checker
%% abc > push x
%% Pushing x
%% abc > agent A = x(y).B y
%% Agent A is defined.
%% abc > show A
%% x(x0). B x x0
%% abc > push y
%% Pushing y
%% abc > agent B = 'x.A
%% Agent B is defined.
%% abc > show B
%% 'x. A x
%% abc > pop
%% Popping at most 1 variables.
%% abc > implicit
%% Implicit variables are x
%% abc > show A
%% x(x0). B x x0
%% abc > show B
%% (\x0) 'x. A x
%% abc > show A
%% (\x0) x0(x1). B x0 x1
%% abc > show B
%% (\x0, x1) 'x0. A x0
%% \end{verbatim}

%% Now, we redo the same actions, to illustrate the use of \textbf{clear}.
%% \begin{verbatim}
%% abc > push x
%% Pushing x
%% abc > agent A = x(y).B y
%% Agent A is defined.
%% abc > show A
%% x(x0). B x0
%% \end{verbatim}

%% We see here that the behaviour is not the one we want. Indeed, now $x$
%% is not an implicit parameter of $B$ since $B$ is already defined and
%% the implicit context of $B$ is now empty. So, we have to clear the
%% definition of $B$ before redoing the same commands.
%% \begin{verbatim}
%% abc > clear B
%% Clearing B
%% abc > agent A = x(y).B y
%% Agent A is defined.
%% abc > show A
%% x(x0). B x x0
%% \end{verbatim}

%% And then, we can continue as before.

\bibliographystyle{alpha}
\bibliography{biblio}
\end{document}