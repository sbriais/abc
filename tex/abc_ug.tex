\documentclass{article}
\usepackage[latin1]{inputenc}
\usepackage{process}

\newcommand{\LAMBDA}[2]{(\lambda {#1}){#2}}

\newenvironment{bnf}{%
  \newif\ifbnftab%
  \bnftabfalse%
  \let\bnfnewline\\%
  \def\IS{%
    &$::=$&%
    \bnftabtrue}
  \def\OR{%
    \ifbnftab%
      \vrule~%
    \else%
      \expandafter\bnfORline%
    \fi%
    \bnftabtrue}%
  \def\bnfORline{&\hfill\vrule&}%
  \def\\{%
    \bnfnewline%
    \bnftabfalse}%
  \def\BRK{\cr&&}%
  \sffamily%
  \par\medskip
  \begin{tabular}{rcll}}{%
  \end{tabular}%
  \par\medskip}


\author{\textsc{S�bastien BRIAIS}}
\title{The ABC User's Guide}

\begin{document}
\maketitle

\begin{abstract}
  We describe in this document how to use our tool called Another
  Bisimilarity Checker (ABC). The ABC is a tool that checks open
  equivalence between terms of \picalculus. Open bisimulation was
  first defined by Sangiorgi \cite{sangiorgi:theory-bisimulation}. The
  ABC implements the equivalence checker of the Mobility Workbench
  \cite{victor:verification-tool}.
\end{abstract}

\section{Input syntax: \picalculus terms}
\subsection{The \picalculus}
We assume a countably infinite set of names denoted by
$a,b,x,v,\ldots$ and a disjoint set of agent identifiers denoted by
$X$. The syntax of \picalculus terms is described by the following grammar:
\begin{bnf}
$P,Q$ \IS $\NIL$ 
      \OR $\PAR{P}{Q}$ 
      \OR $\SUM{P}{Q}$ 
      \OR $\PREFIX{\alpha}{P}$
      \OR $\NU{\TUPLE{x}}{P}$
      \OR $\APPLY{X}{\TUPLE{v}}$
      \OR $\MATCH{a}{b}{P}$\\

$\alpha$ \IS $\INPUT{a}{\TUPLE{x}}$ 
         \OR $\OUTPUT{}{a}{\TUPLE{v}}$ 
         \OR $\TAU$
\end{bnf}
As usual, we write $\TUPLE{v}$ as a shortcut for $v_1,\ldots,v_n$ for some $n$.

Sometimes, \picalculus syntax is presented in terms of concretions and
abstractions. A concretion is mainly something like $\CONC{\TUPLE{v}}{C}$
where $C$ is a process or a concretion. An abstraction is mainly
something like $\LAMBDA{\TUPLE{x}}{A}$ where $A$ is a process or an
abstraction. In this setting, prefixes are either $\TAU$, an input
name $\INPUT{a}{}$ or an output name $\OUTPUT{}{a}{}$. The presentation
in terms of concretions and abstractions make the labelled-transition
semantics easier to write.
It is easy to convert classic syntax to this one. Indeed, we have
$$
\begin{array}{rcl}
\PREFIX{\INPUT{a}{\TUPLE{x}}}{P} & \equiv & \PREFIX{\INPUT{a}{}}{\LAMBDA{\TUPLE{x}}{P}} \\
\PREFIX{\OUTPUT{}{a}{\TUPLE{v}}}{P} & \equiv & \PREFIX{\OUTPUT{}{a}{}}{\CONC{\TUPLE{v}}{P}}
\end{array}
$$

The \picalculus terms accepted by the ABC are the one described by the
previous grammar. It also allows the user to enter a \picalculus
term written in terms of abstractions and concretions.


The ABC also allows to enter a term partially applied to a variable. This makes sense
only if the term is an abstraction.
\subsection{Input syntax}
\begin{tabular}{cc}
\begin{bnf}
agent \IS factor\\
      \OR factor \verb:"+": agent\\
\\
factor \IS term \\
       \OR term \verb:"|": factor\\
\\
term \IS \verb:"0": \\
      \OR ident params \\
      \OR prefix \verb:".": term \\
      \OR \verb:"[": vars \verb:"]": term \\
      \OR \verb:"[": var \verb:"=": var \verb:"]": term \\
      \OR \verb:"(": \verb:"\": vars \verb:")": term \\
      \OR \verb:"(": \verb:"^": vars \verb:")": term \\
      \OR term var \\
      \OR \verb:"(": agent \verb:"):\\
\end{bnf}
&
\begin{bnf}
vars \IS var \\
     \OR var \verb:",": vars \\
\\
prefix \IS \verb:"t": \\
       \OR var params \\
       \OR \verb:"'": var oparams \\
\\
params \IS $\epsilon$ \\
       \OR \verb:"(": vars \verb:")": \\
\\
oparams \IS $\epsilon$ \\
       \OR \verb:"<": vars \verb:">": \\
\\
var \IS [a-z][a-zA-Z\_/.0-9]* \\
\\
ident \IS [A-Z][a-zA-Z\_/.0-9]* \\
\end{bnf}
\end{tabular}

\paragraph{Remark:} Actually, it is optional to write the nil process after an action.

\section{Commands of the ABC}
\subsection{Agent definition}
\label{ss:agentdef}
\paragraph{Syntax}\mbox{}\\
\textbf{agent} {\sffamily ident params = agent}


Note that only \emph{closed agents}\footnote{for efficiency reasons, see also subsection \ref{ss:context}} are allowed (i.e. $\mathrm{fn}$({\sffamily agent}) $\subset${\sffamily params}).

\subsection{Checking (open) equivalence}
\subsubsection{Strong equivalence}
\paragraph{Syntax}\mbox{}\\
\textbf{eq} {\sffamily agent1 agent2}\\
\textbf{eqd} {\sffamily (vars) agent1 agent2}


The command \textbf{eq} checks strong equivalence between {\sffamily agent1} and {\sffamily agent2}. The command \textbf{eqd} takes into account the distinction such that each variable in {\sffamily vars} is distinct from all other free names in {\sffamily agent1} and {\sffamily agent2}.

\subsubsection{Weak equivalence}
\paragraph{Syntax}\mbox{}\\
\textbf{weq} {\sffamily agent1 agent2}\\
\textbf{weqd} {\sffamily (vars) agent1 agent2}


Similar to strong equivalence.

\subsection{Viewing agents}
\paragraph{Syntax}\mbox{}\\
\textbf{print} {\sffamily agent}\\
\textbf{show} {\sffamily agent}


The command \textbf{print} pretty-prints the given {\sffamily agent} whereas the command \textbf{show} pretty-prints the standard form of the given {\sffamily agent}.

\subsection{Exploring the behaviour of an agent}
\paragraph{Syntax}\mbox{}\\
\textbf{step} {\sffamily agent}


The command \textbf{step} enters in an interactive mode in which it is possible to view and simulate the (strong) commitments of the given {\sffamily agent}.

\subsection{Setting memory usage}
\paragraph{Syntax}\mbox{}\\
\textbf{scale} integer\\
\textbf{value} integer

The command \textbf{scale} fixes the scale for setting the memory usage. The command \textbf{value} sets the memory usage to $\frac{\textbf{value}}{\textbf{scale}}$. This rational number represents the frequency for caching computed data.

\subsection{Implicit context}
\label{ss:context}
We found annoying to not allow open terms in agent definition (as
mentioned in subsection \ref{ss:agentdef}). So, we added the notion of
semi-explicit context that we call implicit context. The idea is that
a context is a list of variables that are implicit in agent definition
and agent application. More formally, there is a \emph{current
  implicit context}, and each agent has an \emph{agent implicit
  context}. If an agent is not defined, then its implicit context is
the current one.  So, when an agent is defined, its context is the
current one. It is possible to push or pop variables in the current
implicit context.  When one pops variables out of the current context,
these variables are also popped out of all the agent implicit
contexts.
\paragraph{Syntax}\mbox{}\\
\textbf{push} {\sffamily var$_1$ ... var$_n$}

This command pushes {\sffamily var$_1$ ... var$_n$} into the
current context.

\mbox{}\\
\textbf{pop} [$n$]

This command pops $n$ variables out of the current context ($n$ is a
positive integer, if $n$ is not specified, then $1$ is the value by
default).

\mbox{}\\
\textbf{implicit}

This command shows the current implicit context.

\mbox{}\\
\textbf{clear} {\sffamily ident$_1$ ... ident$_n$}

This command undefines the agent {\sffamily ident$_1$ ... ident$_n$}
if they were defined.


\subsection{Other commands}
\subsubsection{Help}
\paragraph{Syntax}\mbox{}\\
\textbf{help}

The command \textbf{help} give the user some help.

\subsubsection{Resetting the ABC}
\paragraph{Syntax}\mbox{}\\
\textbf{reset}

This command clears all the memory of the ABC.

\subsubsection{Loading a file}
\paragraph{Syntax}\mbox{}\\
\textbf{load} {\sffamily filename}

This command loads the given file in the ABC.

{\sffamily filename} $\in$ {\sffamily var}$\cup${\sffamily ident}.

\subsubsection{Exiting}
\paragraph{Syntax}\mbox{}\\
\textbf{exit}

\textbf{exit} makes you exit the ABC.


\section{Examples}
\subsection{A beginner's example}
Let us start the ABC.
\begin{verbatim}
bash-2.05b$ ./abc.opt
Welcome to Another Bisimulation Checker
\end{verbatim}
Let us define agents $\FUN{A}$ and $\FUN{B}$ with 
$$
\begin{array}{rcl}
\APPLY{A}{x,y} & \EQDEF & \PAR{\PREFIX{\INPUT{x}{}}{\NIL}}{\PREFIX{\OUTPUT{}{y}{}}{\NIL}} \\
\APPLY{B}{x,y} & \EQDEF & \SUM{\PREFIX{\INPUT{x}{}}{\PREFIX{\OUTPUT{}{y}{}}{\NIL}}}{\PREFIX{\OUTPUT{}{y}{}}{\PREFIX{\INPUT{x}{}}{\NIL}}}
\end{array}
$$
\begin{verbatim}
abc > agent A(x,y) = x.0 | 'y.0
Agent A is defined.
abc > agent B(x,y) = x.'y.0 + 'y.x.0
Agent B is defined.
\end{verbatim}
It is well known that $\APPLY{A}{x,y} \not\sim^{\mathrm{o}}_{\EMPTY} \APPLY{B}{x,y}$\footnote{$\sim^{\mathrm{o}}_D$ is the open bisimilarity with the set of distinctions $D$.}.
\begin{verbatim}
abc > eq A(x,y) B(x,y)
The two agents are not strongly related (2).
Do you want to see a trace of execution (yes/no) ? yes
trace of A x y
 -t-> 0
trace of B x y
 

\end{verbatim}
Indeed, $\APPLY{A}{x,y}$ can perform a $\TAU$ commitment that $\APPLY{B}{x,y}$ can't (as shown in the trace).
\begin{verbatim}
abc > step A(x,y)
1: { x=y }  =>  A x y --t-> 0

2: {  }  =>  A x y --x-> 'y.0

3: {  }  =>  A x y --'y-> x.0

Please choose a commitment (between 1 and 3) or 0 to exit: 1
no more commitments
abc > step B(x,y)
1: {  }  =>  B x y --x-> 'y.0

2: {  }  =>  B x y --'y-> x.0

Please choose a commitment (between 1 and 2) or 0 to exit: 1
1: {  }  =>  'y.0 --'y-> 0

Please choose a commitment (between 1 and 1) or 0 to exit: 1
no more commitments
\end{verbatim}

\paragraph{Remark:}
We see above that the commitments are shown numbered. Each commitment has the form: 
\begin{verbatim}
a set of conditions => agent --action--> agent
\end{verbatim}
The set of conditions is the one that should be satisfied in order for
the commitment to take place.

\bigskip

Thus, it is easy to see that $\APPLY{A}{x,y} \sim^{\mathrm{o}}_{\SET{x \not= y}} \APPLY{B}{x,y}$.
\begin{verbatim}
abc > eqd (x,y) A(x,y) B(x,y)
The two agents are strongly related (3).
Do you want to see the core of the bisimulation (yes/no) ? yes
{
  (
    x.0
    {
      (x, y)
    }
    x.0
  )

  (
    'y.0
    {
      (x, y)
    }
    'y.0
  )

  (
    ('y.x.0 + x.'y.0)
    {
      (x, y)
    }
    ('y.0 | x.0)
  )
}
\end{verbatim}

\paragraph{Remark:}
What we call the \emph{core} of a bisimulation is a ternary relation
between an agent, a set of distinctions and an other agent such that
its symmetric closure plus the identity relation is a bisimulation.

\bigskip

Moreover, if we define an agent $\FUN{C}$ such that
$$
\begin{array}{rcl}
\APPLY{C}{x,y} & \EQDEF & \SUM{\APPLY{B}{x,y}}{\MATCH{x}{y}{\PREFIX{\TAU}{\NIL}}}
\end{array}
$$
\begin{verbatim}
abc > agent C(x,y) = B(x,y) + [x=y]t.0
Agent C is defined.
\end{verbatim}
We have $\APPLY{A}{x,y} \sim^{\mathrm{o}}_{\EMPTY} \APPLY{C}{x,y}$.
\begin{verbatim}
abc > eq A(x,y) C(x,y)
The two agents are strongly related (4).
Do you want to see the core of the bisimulation (yes/no) ? yes
{
  (
    0
    { }
    0
  )

  (
    x.0
    { }
    x.0
  )

  (
    'y.0
    { }
    'y.0
  )

  (
    ([x=y]t.0 + 'y.x.0 + x.'y.0)
    { }
    ('y.0 | x.0)
  )
}
\end{verbatim}
End of this example.
\begin{verbatim}
abc > exit
bash-2.05b$
\end{verbatim}

\subsection{Usage of the implicit context}
\subsubsection{Basic example}
At the beginning, the current implicit context is empty.
\begin{verbatim}
bash-2.05b$ ./abc
Welcome to Another Bisimulation Checker
abc > implicit
No implicit variables.
\end{verbatim}

Assume that we want to define an agent $\APPLY{A}{x} \EQDEF
\SUM{\PREFIX{\INPUT{x}{}}{\NIL}}{\PREFIX{\TAU}{\APPLY{A}{x}}}$, but with using the concept of implicit
context. 


We first push $x$ into the current context.
\begin{verbatim}
abc > push x
Pushing x
abc > implicit
Implicit variables are x
\end{verbatim}

We then define our agent $A$.
\begin{verbatim}
abc > agent A = x.0 + t.A
Agent A is defined.
\end{verbatim}

Let us see what is the standard form of $A$.
\begin{verbatim}
abc > show A
( x. 0 +  t. A x )
\end{verbatim}
Note that when we write $A$, it really means $A\ x$, since the implicit
context of $A$ is $x$.


Now, we push another name $y$, see the standard form of $A$ and then
we pop the two names we had pushed and see again the standard form of
$A$.
\begin{verbatim}
abc > push y
Pushing y
abc > implicit
Implicit variables are x y
abc > show A
( x. 0 +  t. A x )
abc > pop 2
Popping at most 2 variables.
abc > implicit
No implicit variables.
abc > show A
(\x0) ( x0. 0 +  t. A x0 )
abc > exit
bash-2.05b$
\end{verbatim}

\subsubsection{Advanced example}
One difficulty that appears with the concept of implicit context is
the behavior when there is mutually recursive agent definitions.


In the following, we define two mutually recursive agents $A$ and $B$,
with $\APPLY{A}{x} = \PREFIX{\INPUT{x}{y}}{\APPLY{B}{x,y}}$ and
$\APPLY{B}{x,y} = \PREFIX{\OUTPUT{}{x}{}}{\APPLY{A}{x}}$.
\begin{verbatim}
bash-2.05b$ ./abc
Welcome to Another Bisimulation Checker
abc > push x
Pushing x
abc > agent A = x(y).B y
Agent A is defined.
abc > show A
x(x0). B x x0
abc > push y
Pushing y
abc > agent B = 'x.A
Agent B is defined.
abc > show B
'x. A x
abc > pop
Popping at most 1 variables.
abc > implicit
Implicit variables are x
abc > show A
x(x0). B x x0
abc > show B
(\x0) 'x. A x
abc > show A
(\x0) x0(x1). B x0 x1
abc > show B
(\x0, x1) 'x0. A x0
\end{verbatim}

Now, we redo the same actions, to illustrate the use of \textbf{clear}.
\begin{verbatim}
abc > push x
Pushing x
abc > agent A = x(y).B y
Agent A is defined.
abc > show A
x(x0). B x0
\end{verbatim}

We see here that the behaviour is not the one we want. Indeed, now $x$
is not an implicit parameter of $B$ since $B$ is already defined and
the implicit context of $B$ is now empty. So, we have to clear the
definition of $B$ before redoing the same commands.
\begin{verbatim}
abc > clear B
Clearing B
abc > agent A = x(y).B y
Agent A is defined.
abc > show A
x(x0). B x x0
\end{verbatim}

And then, we can continue as before.

\bibliographystyle{alpha}
\bibliography{biblio}
\end{document}